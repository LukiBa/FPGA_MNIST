\subsection{conv2d}

\begin{figure}[h]
	\centering
	\includesvg[width=0.7\textwidth]{img/inkscape/conv2d.svg}
	\caption[Conv2d block diagram.]{Conv2d block diagram. For each output channel a conv\_channel module is used. $k$ indicates the number of output channels.}
	\label{FIG:conv2d}
\end{figure}
Figure \ref{FIG:conv2d} shows the block diagram of a conv2d module. It uses $k$ conv\_channel modules to realise $k$ output channels. 
All conv\_channel modules get the same input vector $X_{c_i}$. All conv\_channel modules
and the two conv2d modules are automatically generated by a Python script.
\subsubsection{Interface}
\begin{itemize}
	\item Input interface connected to shift register, which consists of a $n \cdot 3 \times 3$ vector of values of length BIT\_WIDTH\_IN, in which $n$ is the number of input channels.
	\item Output interface connected to the pooling layer, which is a vector of $m$ values of length BIT\_WIDTH\_OUT, in which $m$ is the number of output channels.
\end{itemize}
Both input and output interfaces have ready, last and valid signals to control the flow of data.
\subsubsection{Parameter}
\begin{itemize}
 	\item BIT\_WIDTH\_IN : integer
 	\item BIT\_WIDTH\_OUT : integer
 	\item INPUT\_CHANNELS: integer
 	\item OUTPUT\_CHANNELS: integer
\end{itemize}