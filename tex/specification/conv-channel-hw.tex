\subsection{conv\_channel}

\begin{figure}[h]
	\centering
	\includesvg[width=0.7\textwidth]{img/inkscape/conv-channel.svg}
	\caption[conv\_channel block diagram.]{conv\_channel block diagram. For each input channel a kernel\_3x3 module is used. $n$ indicates the number of input channels.}
	\label{FIG:conv-channel}
\end{figure}
Figure \ref{FIG:conv-channel} shows the block diagram of a conv\_channel module. It uses $n$ kernel\_3x3 modules to realise $n$ input channels. 
All kernel\_3x3 modules get a different input vector $X_{c_{i1}}$ to $X_{c_{in}}$ which are $3 \times 3$ input matrices. All kernel outputs are summed up
to one final value of length BIT\_WIDTH\_OUT.
\subsubsection{Interface}
\begin{itemize}
	\item Input interface, same as conv2d.
	\item Output interface connected to the pooling layer, which is a value of length BIT\_WIDTH\_OUT.
\end{itemize}
\subsubsection{Parameter}
\begin{itemize}
 	\item BIT\_WIDTH\_IN : integer
 	\item KERNEL\_WIDTH\_OUT : integer, output bit width of the kernel\_3x3 module
 	\item BIT\_WIDTH\_OUT: integer
 	\item N: integer, number of kernels
 	\item OUTPUT\_MSB: integer, defines which of the $n$=BIT\_WIDTH\_OUT bits is the most significant bit
 	\item BIAS: integer, currently unused as bias seems to not be very important in the convolutional layers
\end{itemize}