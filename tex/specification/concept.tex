\section{Concept}
\begin{figure}[h]
	\centering
	\includesvg[width=\textwidth]{img/inkscape/NN-concept.svg}
	\caption[Top-Level concept.]{Top-Level concept}
	\label{FIG:concept}
\end{figure}
\todo{Maybe add an additional Input Layer which is responsible to communicate with the DMA and converts the data from 32 bit to 8 bit and sends it to the memory controller}
\noindent
Figure \ref{FIG:concept} shows the Concept of implementing an FPGA-based hardware accelerator for handwritten digit recognition. It shows that the main components of the concepts are a Zedboard in combination with a remote PC or server. The handwritten digit recognition is performed by the Zedboard while the remote PC is used for training the network, for sending the image data to the Zedboard and for receiving the computed results.  The Zedboard includes a Zynq-7000 FPGA and provides various interfaces. \\
The neural network is implemented in the programmable logic part of the Zynq-7000. It is pre-trained using the remote PC, therefore only the inference of the neural network is implemented in hardware. \\
In order to train the network with the same bit resolution as implemented in the hardware, a software counterpart of the hardware is implemented in a PC using python. 
Based on the weights calculated by the python script a bitstream for the hardware is generated. This brings the benefit that for the convolutional layer constant multiplier can be used, since the weights of convolutional layer kernels are constant. For the dense layer it is not possible to implement the weights in a constant multiplier because in a dense layer each connection of a neuron requires a different weight, which would result in a huge amount of required constant multipliers. Therefore the weights for the dense layer have to be stored in a ROM inside the FPGA.   \\
